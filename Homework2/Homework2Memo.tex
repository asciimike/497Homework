\documentclass[letterpaper,10pt]{texMemo}

\usepackage{graphicx}
\usepackage{listings}


\memoto{Dr. Mark Yoder}
\memofrom{Michael McDonald}
\memosubject{Problem Set 2 (10, 10a, 11) Memo}
\memodate{\today}
\logo{\includegraphics[width=0.5\textwidth]{RoseLogo.png}}

\begin{document}
\maketitle

\section{Weekly Summary}

\noindent This week was devoted to GPIO, both using the Linux file system as
well as using C. 

\newpage
\subsection{Exercise Summary}

\subsubsection{Exercise 10}

\noindent Reading a switch can be accomplished by the following commands
\begin{verbatim}
 host$ sys/class/gpio: echo XX > export # add pin XX
 host$ sys/class/gpio: cd gpioXX # go into the gpio directory
 host$ sys/class/gpio/gpioXX: echo in > direction # make the pin an input
 host$ sys/class/gpio/gpioXX: cat value # read the value in from the port
\end{verbatim}

\begin{enumerate}
 \item What's the min and max voltage? Min: -100mV, Max 3.41V.
 \item What period is it? The period varies from 250ms to 260ms, with
occasional jumps as high as 350ms or 400ms.
 \item How close is it to 200ms? Off by about 50ms
 \item Why do they differ? Because the OS schedules it to run and occasionally
gives priority to other processes.
 \item Run htop and see how much processor you are using. 1-2\% of the CPU, and
less than 0.25\% of the RAM.
 \item Try different values for the sleep time (2nd argument). What's the
shortest
period you can get? Make a table of the values you try and the corresponding
period and processor usage. THe shortest period I can get is around 70 to 80ms.
 \item How stable is the period? The period is not very stable, it usually
varies by 10 to 20ms, and occasionally jumps 100 to 150ms higher.
 \item Try launching something like mplayer. How stable is the period? I
launched gedit, and it was very unstable when the app was launching, but it
stabilized close to what we saw earlier when it was just running in the
background. When it was in use, however, the period was much less stable, and
longer (300-400ms).
 \item Try cleaning up togglegpio.sh and removing unneeded lines. Does it impact
the period? Cleaning it up shaved off about 20-30ms, which is 10\% or so, so it
did a pretty good job in my opinion.
 \item togglegpio uses bash (first line in file). Try using sh. Is the period
shorter? Running it with sh seems to take about the same amount of time, though
it seems to be a little more stable.
 \item What's the shortest period you can get? sh could run much faster than
bash, and I saw periods as short 25-30ms.
\end{enumerate}

\begin{center}
\begin{tabular}{| l | l | l |}
\hline
\textbf{Time (s)} & \textbf{Period (ms)} & \textbf{CPU Load (\%)}\\
\hline
0.1 & 250-260 & 1-2\\
\hline
0.05 & 150-170 & 3-4\\
\hline
0.01 & 70-150  & 5-6\\ 
\hline
0.01 & 70-200 & 6-8\\
\hline
\end{tabular}
\end{center}


\newpage
% Modify togglegpio.sh (call it toggleLED) to toggle the on-board LEDs. Can you
% get the LED to appear to dim by changing the duty cycle of the toggling?
\lstinputlisting{toggleLED.sh}

\newpage
% Display a count of how many times button has been pressed
\lstinputlisting{countgpio.sh}

\newpage
% Write a shell script that copies the value of gpio pin 7 (your switch) to gpio
% pin 60 (Your LED). How much CPU time does it take? What's the delay from the
% time the input changes until the output changes? How constant is the delay?
% 
% You may need to get a function generator and set it to a square wave (0 to 3.3V)
% and attach it to pin 7 and attached an oscilloscope to pin 60. 
\lstinputlisting{controlLED.sh}

\noindent The controlLED shell script uses 6-8\% of the CPU, and less than
0.25\% of the memory, according to htop. The delay is about 20ms, and it remains
fairly constant.

\newpage
\subsubsection{Exercise 10a}
\noindent The min and max value that I've been able to read are 0 to 1800,
and 12 bits is 4096. Therefore, it is reasonable to assume that AIN0-AIN6
output a 0.000 to 1.800 volt signal, where a difference by one is a 1mV
difference.

\lstinputlisting{continuousADC.sh}

\subsubsection{Exercise 11}

\newpage

\begin{center}
\begin{tabular}{|p{4cm}|l|l|p{8cm}|}
\hline
\textbf{Objective} & \textbf{Exercise} &\textbf{Status} & \textbf{Notes}\\
\hline
& & & \\
\hline
\end{tabular}
\end{center}

\noindent The main points I took away from these exercises were the following:
\begin{itemize}
 \item
\end{itemize}

\end{document}
